
\section{Introduction} \label{sec:Intro}
%%% INTRODUCTION OF TOPIC %%%
The ever-increasing availability of macroeconomic data and recent developments in the multivariate time series analysis for panels have popularized panel vector autoregressive methods in applied econometric research. The \pkg{pvars} package summarizes a toolkit for the empirical analysis on panels in the narrow sense, where the same time series variables are repetitively observed across several individual entities such as countries. This three-dimensional data structure thus contrasts with panels in a wider sense, where time series are just arranged along the same periods. Although \pkg{pvars}' application is not bounded to financial or macroeconometric research, it addresses the data properties typically found in macroeconomic panels, namely (i) \textit{mostly heterogeneous} (ii) \textit{endogenous interactions} between (iii) potentially \textit{non-stationary} variables. Since (iv) the \textit{time dimension is distinctively larger} than the cross-section, the time series are prone to (v) a \textit{complex deterministic term} with structural breaks in their mean and linear trend. The individual entities are usually subject to (vi) \textit{cross-sectional dependencies}.\footnote{See \citet{Pedroni2019} for an intuition and discussion of these panel properties and an overview on recent developments of single-equation and system-based methods. The data characteristics can be found in the applied econometrics of climate \citep{Pretis2020}, energy \citep{SmythNarayan2015}, and growth linked to financial development \citep{ChristopoulosTsionas2004} or public capital \citep{EmptingHerwartz2021b}.}

%%% APPROACHES FOR HETEROGENEOUS PANELS %%%
In order to deal with these data properties, the econometric literature extends individual time series methods by the cross-sectional dimension under selective pooling assumptions. For example, the panel methods implemented by the \proglang{Stata} commands \pkg{xtcointtest} \citep{StataCorp2019}, \pkg{xtwest} \citep{PersynWesterlund2008}, and \pkg{xtpmg} \citep{BlackburneFrank2007} originate from the single-equation framework comprising residual-based cointegration tests and autoregressive distributive lag (ARDL) models in error-correction representation. 
In contrast, \pkg{pvars} relies on vector autoregressive (VAR) models as a system of equations. This approach has the advantage of avoiding restrictive exogeneity assumptions on the variables and modeling their dynamics and interactions explicitly.
The model in vector error correction representation (VECM) can further accommodate multiple long-run relations and different types of deterministic regressors within the cointegration. Particularly the cointegration relations are reasonable candidates for a panel-wide pooling, while short-run dynamics are usually assumed to differ across individuals. 

%%% NICHE IN THE ECOSYSTEM OF R-PACKAGES %%%
The synthesis of VAR models and methods for heterogeneous panels has not been implemented yet, but many packages for \proglang{R} \citeyearpar{RCore2020} can already contribute specifications results and individual counter-checks. Hence, they may be subsumed into two pillars for \pkg{pvars}: Firstly, the \pkg{vars}-ecosystem with \pkg{urca} \citep{Pfaff2008}, \pkg{vars} \citep{Pfaff2008a}, and \pkg{svars} \citep{LangeEtAl_fc} covers individual VAR methods such as VECM and structural identification. The second pillar are the single-equation panel methods from \pkg{plm} \citep{CroissantMillo2008} such as dynamic panel regression models and panel unit root tests. Univariate panel time series can be tested for common or idiosyncratic non-stationarity using the more specialized package \pkg{PANICr} \citep{Bronder2016}. Like \pkg{pvars}, the \proglang{R}-package \pkg{panelvar} \citep{SigmundFerstl2019} and its \proglang{Stata}-equivalent \pkg{pvar} \citep{AbrigoLove2016} rest on the two pillars. However, they focus on stationary panels that contain more individuals than periods ($ N > T $), and the implemented estimators rely on slope coefficients that are homogeneous across all individuals in the PVAR model. Hence, they rather suit microeconomic applications, and their primary task is to deal with the \textit{Nickell bias} \citeyearpar{Nickell1981}.\footnote{See also \citet{CanovaCiccarelli2013} for a survey on panel VAR models.} 

%%% OBJECTIVE: CONTRIBUTION OF THIS PACKAGE %%%
The objective of \pkg{pvars} is to close the gap in the \pkg{vars}-ecosystem and provide a seamless implementation of the fruitful panel VAR methodology. For this, three fields of VAR applications are integrated into \pkg{pvars}, namely  panel estimation, cointegration testing, and structural identification. The implemented methods particularly noteworthy for each application field are (1) panel VECM with \textit{pooled cointegrating vectors} from Breitung's \citeyearpar{Breitung2005} two-step estimator. (2) The new panel cointegration rank tests by Arsova and Örsal \citeyearpar{ArsovaOersal2017,ArsovaOersal2018} respect cross-sectional dependencies stemming from common factors. \citet{ArsovaOersal2020} consider also structural breaks in the deterministic term. (3) Data-driven identification methods based on \textit{independent component analysis} (ICA) are extended for panels by \citet{CalhounEtAl2002} and \cite{Herwartz2017}.

%%% STRUCTURE OF THE ARTICLE %%%
The remainder of this article is structured as follows: Section~\ref{sec:Review} is a review on the econometric methods which are relevant for using this \proglang{R}-package. Section~\ref{sec:Implementation} highlights the available functions in \pkg{pvars}, their implementation, and their library of auxiliary functions. In Section~\ref{sec:Illustration}, we demonstrate two empirical applications of \pkg{pvars} to exemplary data from the reviewed articles. Finally, Section~\ref{sec:Summary} summarizes this article. 


